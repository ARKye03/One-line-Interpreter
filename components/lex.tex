\begin{center}
    \centering
    \section{Lexer}
\end{center}

\subsection{Lexer Summary}

A Lexer, also known as a lexical analyzer, tokenizer, or scanner, is a program or function that breaks down input code into a sequence of tokens. These tokens are meaningful units of code. 

In the process of parsing programming languages, this task is traditionally split up into two phases: the lexing stage and the parsing stage. The lexer's job is simpler than the parser's. It turns the meaningless string into a flat list of things like ``number literal'', ``string literal'', ``identifier'', or ``operator''. It can also recognize reserved identifiers (``keywords'') and discard whitespace. 

The output of the lexer is used as the input for the parser. The parser then has the much harder job of turning the stream of ``tokens'' produced by the lexer into a parse tree representing the structure of the parsed language. This separation allows the lexer to do its job well and for the parser to work on a simpler, more meaningful input than the raw text.

More info about:
\begin{itemize}
    \item \href{https://dev.to/cad97/what-is-a-lexer-anyway-4kdo}{What's a Lexer}
    \item \href{https://accu.org/journals/overload/26/145/balaam_2510/}{The Lexer - ACCU}
    \item \href{https://www.programmingassignmenthelper.com/blog/lexer-and-parser-in-c/}{Extra blog}
\end{itemize}

\lstset{style=sharpc}
\subsection{Now, my Lexer}
    First I've defined some simple Token types in this ``enum'':
    \begin{lstlisting}
        public enum TokenType
        {
            FLinq, //Link token for function declaration
            ComparisonOperator,
            Number, // Number
            StringLiteral, // Literally a string
            FunctionDeclaration, // Well, deprecated
            LetKeyword, 
            IfKeyword, 
            ElseKeyword, 
            PrintKeyword, 
            InKeyword,
            FunctionKeyword,
            Operator, // '+','-','*','/','^', '='...
            Punctuation, // '(',')' & ','
            Identifier, //Used for variables name and functions value
            EOL, //';'
            EOF, //Not yet
            Semicolon, //Idk man/woman
            Separator // '@', and in a future, "||", "&&"...
        }
    \end{lstlisting}
    Basically, my Tokens, have a few properties:
    \begin{lstlisting}
        public TokenType type; // From enum
        public string value; // Text of the token
        public int line; // Line position, currently 1, always
        public int column; // Column position
    \end{lstlisting}
    \texttt{Note:} ToString override is there to test tokens only.

    \newpage
    \begin{center}
        \centering
        Some functions of my lexer
    \end{center}
\begin{itemize}
    \item One char to the right:
    \begin{lstlisting}
    private void advance()
    {
        pos++;
        column++;
        if (pos > text.Length - 1)
        {
            current_char = '\0';
        }
        else
        {
            current_char = text[pos];
        }
    }
    \end{lstlisting}
    \item Jumps to next non-blank char:
    \begin{lstlisting}
    private void skip_whitespace()
    {
        while (current_char != '\0' && char.IsWhiteSpace(current_char))
        {
            advance();
        }
    }
    \end{lstlisting}
    \item Takes a look to next token, without taking it:
    \begin{lstlisting}
    private char peek()
    {
        var peek_pos = pos + 1;
        if (peek_pos >= text.Length)
        {
            return '\0';
        }
        else
        {
            return text[peek_pos];
        }
    }
    \end{lstlisting}
    \item Drop a token to be used later:
    \begin{lstlisting}
    public void unget_token(Token token)
    {
        readTokens.Insert(0, token);
    }
    \end{lstlisting}
\end{itemize}

\begin{center}
    \centering
    The main function
\end{center}
The \hbox{get\_next\_token{()}} method it's actually a little bit big, it used several times in the `Interpreter' itself, since it is in charge of tokenizing/lexing the expression as it is parsed.